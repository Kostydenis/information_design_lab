%!TEX root = main.tex
\section{Постановка задачи}

Имеется $N$ различных судов, которые могут осуществлять перевозку грузов по регулярным линиям. Суда различных типов при эксплуатации на той или иной линии имеют различные характеристики. В частности различные суда имеют различное число заходов в промежуточные порты для пополнения запасов пресной воды и топлива.

На некоторых линиях используется не полностью коммерческая грузоподъемность судов, а суда некоторых типов вообще не могут быть использованы на некоторых линиях.

Все эти причины приводят к тому, что эксплуатационные расходы, приходящиеся на каждых используемых на линии оказываются различными. Исходя из данных о себестоимости грузокилометров и коммерческой загрузки судов на каждой линии устанавливается:

\begin{itemize*}
	\item $a_{ij}$ --- месячный объем перевозки одним судом $j$ типа груза на $i$ регулярной линии
	\item $c_{ij}$ --- месячный эксплуатационный расход на одно судно $j$ типа на $i$ линии.
\end{itemize*}

Предполагается, что известен требуемый месячный объем перевозки по каждой регулярной линии. Известно, также число судов $j$ типа.

Требуется составить такой план распределения $N$ судов по регулярным линиям, который обеспечивает минимум суммарных эксплуатационных расходов.

Решить поставленную задачу методом симплекс-таблиц, основанном на методе полного исключения Гаусса, применяя для нахождения начального допустимого базисного решения метод Данцига. При этом необходимо распределить весь парк $N$ судов по регулярным линиям.