\section{Описание приложения, формирующего отчет}

Для выполнения работы был выбран язык программирования Python 3.

По ходу выполнения программы, все данные, которые необходимы в отчете записывали в текстовый файл \texttt{solution.tex} встроенными методами. Пример приведен в листинге~\ref{lst:write}: строка 1 --- открытие файла для записи, строка 2 --- запись заголовка в файл.

\lstcode{codes/write.py}{0.68}{\footnotesize}{Пример записи текста в файл}{write}

Также были созданы дополнительные функции для преобразования в выходной формат и записи в файл таблиц (\ref{lst:writetbl}) и формул (\ref{lst:writeeq})

\lstcode{codes/writetbl.py}{0.68}{\footnotesize}{Функция записи таблицы в выходной файл}{writetbl}
\lstcode{codes/writeeq.py}{0.68}{\footnotesize}{Функция записи формулы в выходной файл}{writeeq}

Таким образом формирование формул будет выглядеть, как показано в листинге~\ref{lst:eqexample}.

\lstcode{codes/eqexample.py}{0.68}{\footnotesize}{Пример формирования формулы}{eqexample}

Перед завершением, приложение выполняет запускает компилятор (\texttt{pdflatex}) (листинг~\ref{lst:compile}), после чего в текущей директории создается .pdf-файл отчета.

\lstcode{codes/compile.py}{0.68}{\footnotesize}{Запуск компилятора}{compile}